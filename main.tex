%%%%%%%%%%%%%%%%%%%%%%%%%%%%%%%%%%%%%%%%%%%%%%%%%%%%%%%%%%%%%%%%%%%%%
%% This is a "template" model document for submission to the
%% American Chemical Society (ACS).
%%
%% The guidance here contains information about how you may wish to
%% modify it to match the requirements of various ACS journals. The
%% ACS do *not* typeset accepted articles using LaTeX, so there is
%% no specific class required.
%%
%% This template deliberately does *not* seek to reproduce
%% the layout of the typeset journal: this is explicitly not
%% required by the ACS for LaTeX submissions.
%%
%% Please report any issues with the template at
%% https://github.com/josephwright/acs-template/issues
%%
%% Released under the Creative Commons 0 license
%% https://creativecommons.org/public-domain/cc0/
%% 
%% Copyight (c) 2025 Joseph Wright
%%%%%%%%%%%%%%%%%%%%%%%%%%%%%%%%%%%%%%%%%%%%%%%%%%%%%%%%%%%%%%%%%%%%%
\documentclass[letterpaper]{article}

%%%%%%%%%%%%%%%%%%%%%%%%%%%%%%%%%%%%%%%%%%%%%%%%%%%%%%%%%%%%%%%%%%%%%
%% Font setup - delete if you are using LuaLaTeX
%%%%%%%%%%%%%%%%%%%%%%%%%%%%%%%%%%%%%%%%%%%%%%%%%%%%%%%%%%%%%%%%%%%%%
\usepackage[T1]{fontenc}

%%%%%%%%%%%%%%%%%%%%%%%%%%%%%%%%%%%%%%%%%%%%%%%%%%%%%%%%%%%%%%%%%%%%%
%% Adjust the margins and allow for line spacing
%%%%%%%%%%%%%%%%%%%%%%%%%%%%%%%%%%%%%%%%%%%%%%%%%%%%%%%%%%%%%%%%%%%%%
\usepackage{geometry}
\geometry{margin = 1in}
\usepackage{setspace}

%%%%%%%%%%%%%%%%%%%%%%%%%%%%%%%%%%%%%%%%%%%%%%%%%%%%%%%%%%%%%%%%%%%%%
%% Reference support
%%
%% The recommended method for producing the reference section is
%% to use biblatex. If you wish to use a classical BibTeX
%% approach, this is easiest to achieve using the achemso package.
%% In that case, you should remove the biblatex lines.
%%%%%%%%%%%%%%%%%%%%%%%%%%%%%%%%%%%%%%%%%%%%%%%%%%%%%%%%%%%%%%%%%%%%%
% You can adjust the printing of DOI, article title, etc. using
% package options, e.g. "doi = true"; see the biblatex manual for
% details of adjusting the number of authors printed, e.g.
% "maxnames = 15" to print no more than 15 authors.
\usepackage[style = chem-acs, articletitle = true]{biblatex}
\DeclareFieldFormat{journaltitle}{\mkbibemph{\printfield{shortjournal}}}
\addbibresource{reference.bib}
% If you are using classical BibTeX, remove the above lines and 
% uncomment:
%\usepackage{achemso}

%%%%%%%%%%%%%%%%%%%%%%%%%%%%%%%%%%%%%%%%%%%%%%%%%%%%%%%%%%%%%%%%%%%%%
%% Graphic inclusion and scheme and chart support
%%%%%%%%%%%%%%%%%%%%%%%%%%%%%%%%%%%%%%%%%%%%%%%%%%%%%%%%%%%%%%%%%%%%%
\usepackage{graphicx}
\usepackage{float}
\newfloat{scheme}{htbp}{los}
\floatname{scheme}{Scheme}
\floatname{chart}{Chart}
\newfloat{graph}{htbp}{loh}

%%%%%%%%%%%%%%%%%%%%%%%%%%%%%%%%%%%%%%%%%%%%%%%%%%%%%%%%%%%%%%%%%%%%%
%% Common support packages
%%%%%%%%%%%%%%%%%%%%%%%%%%%%%%%%%%%%%%%%%%%%%%%%%%%%%%%%%%%%%%%%%%%%%
\usepackage{chemformula} % Formulas using \ch{}
% or
\usepackage[version = 4]{mhchem} % Formulas using \ce{}

%%%%%%%%%%%%%%%%%%%%%%%%%%%%%%%%%%%%%%%%%%%%%%%%%%%%%%%%%%%%%%%%%%%%%
%% Many journals require that sections are unnumbered: this 
%% is activated here
%%%%%%%%%%%%%%%%%%%%%%%%%%%%%%%%%%%%%%%%%%%%%%%%%%%%%%%%%%%%%%%%%%%%%
\setcounter{secnumdepth}{-1}

%%%%%%%%%%%%%%%%%%%%%%%%%%%%%%%%%%%%%%%%%%%%%%%%%%%%%%%%%%%%%%%%%%%%%
%% Place any additional macros here.  Please use \newcommand* where
%% possible, and avoid layout-changing macros (which are not used
%% when typesetting).
%%%%%%%%%%%%%%%%%%%%%%%%%%%%%%%%%%%%%%%%%%%%%%%%%%%%%%%%%%%%%%%%%%%%%
\newcommand*\mycommand[1]{\texttt{\emph{#1}}}
\newcommand{\etal}{{\it et al\/}\ }
%%%%%%%%%%%%%%%%%%%%%%%%%%%%%%%%%%%%%%%%%%%%%%%%%%%%%%%%%%%%%%%%%%%%%
%% Author and title data:
%% the authblk package is currently the simplest way to provide this
%%%%%%%%%%%%%%%%%%%%%%%%%%%%%%%%%%%%%%%%%%%%%%%%%%%%%%%%%%%%%%%%%%%%%
\usepackage{authblk}
\author[1,2]{Yuta Goto}
\author[3]{Yuta Ito}
\author[3]{Hiromu Kudo}
\author[4]{Atsushi Ogura}
\author[1]{Ryo Matsumura*}
\author[1,2]{Fukata Naoki*}
\affil[1]{International Center for Materials Nanoarchitectonics (MANA), National Institute for Materials Science, 1-1 Namiki, Tsukuba, Ibaraki, 305-0044, Japan}
\affil[2]{Graduate School of Science and Technology, University of Tsukuba, 1-1-1 Tennodai, Tsukuba, Ibaraki, 305-8573, Japan}
\affil[3]{School of Science and Technology, Meiji University, 1-1-1 Higashimata, Tama-ku, Kawasaki, Kanagawa, 214-8571, Japan}
\affil[4]{Meiji Renewable Energy Laboratory, Meiji University, 1-1-1 Higashimata, Tama-ku, Kawasaki, Kanagawa, 214-8571, Japan}

\title{Improvement of Light Emission and Wavelength Tunability of n-Ge(Si) Thin Films under High-Speed Continuous-Wave Laser Annealing}
% Use the \date command for email address(s) of corresponding authors
\date{*Email: MATSUMURA.Ryo@nims.go.jp, FUKATA.Naoki@nims.go.jp}

\begin{document}

\maketitle

\begin{abstract}
  % Do not exceed 300 words!
  Germanium is a promising material for optoelectronic devices owing to its quasi-direct bandgap under tensile strain and heavy n-type doping.
  However, achieving electron concentrations above 10$^{20}$ cm$^{-3}$ has been challenging because of the low solid solubility of dopants in Ge.
  Nonequilibrium annealing techniques such as pulsed laser annealing have been explored to overcome this limitation, although the extremely rapid cooling rates restrict crystal growth.
  In this work, we demonstrate electron concentration exceeding 1.0 $\times$ 10$^{20}$ cm$^{-3}$ together with high mobility ($\sim$ 100 cm$^2$/Vs) in Ge films by employing high-speed continuous-wave laser annealing. 
  Owing to the microsecond-scale thermal process, this method simultaneously enables large-grain crystallization and efficient dopant activation, as confirmed by Raman spectroscopy and EBSD analysis.
  Furthermore, the approach allows for high electron concentrations in n-Si$_x$Ge$_{1-x}$ (0 $\leq$ x $\leq$ 0.25).
  Photoluminescence measurements reveal tunable emission wavelengths through the combined effects of Si incorporation and laser annealing.
  These results indicate that the annealing method effectively introduces the in-plane tensile strain, activates dopants, and provides a versatile route for achieving highly doped Ge-based materials for infrared optoelectronic applications.
\end{abstract}

\section*{Keywords}
nonequilibrium growth, germanium, laser anneal, solubility

%%%%%%%%%%%%%%%%%%%%%%%%%%%%%%%%%%%%%%%%%%%%%%%%%%%%%%%%%%%%%%%%%%%%%
%% Start the main part of the manuscript here.
%%%%%%%%%%%%%%%%%%%%%%%%%%%%%%%%%%%%%%%%%%%%%%%%%%%%%%%%%%%%%%%%%%%%%
\section{Introduction}

\section{Results and discussion}
%%%%%% Section : FIG1 %%%%%%

%%%%%% Section : FIG2 %%%%%%

%%%%%% Section : FIG3 %%%%%%

%%%%%% Section : FIG4 %%%%%%

%%%%%% Section : FIG5 %%%%%%


\section{Experimental methods}
\subsection{Sample preparation}

\subsection{Characterization}

\section*{Acknowledgements}

This research was financially supported by JSPS Kakenhi (grant No.23K13370), and the World Premier International Research Center Initiative (WPI-Initiative).
This work was also partially supported by "Advanced Research Infrastructure for Materials and Nanotechnology in Japan (ARIM)", MEXT, Japan, proposal number.

\section*{Supporting information}

The following files are available free of charge.
\begin{itemize}
  \item support\_info.pdf: Chemical composition of n-Si$_x$Ge$_{1-x}$ films measured by AES (Figure S1), Depth profile of Sb atomic concentration of n-Si$_x$Ge$_{1-x}$ (x $=$ 0.11) film measured by SIMS (Figure S2),  Ge-Ge vibrational phonon peak position of i-Ge and n-Ge without capping layer measured by Raman spectroscopy (Figure S3), and Raman spectra of n-Si$_x$Ge$_{1-x}$ films in a wider range of Raman shift (Figure S4).   
\end{itemize}

\printbibliography
%\bibliography{acs-template.bib}

\newpage

% \rule{0.05in}{1.75in}%
% \begin{minipage}[b][1.75in]{3.25in}
%   \sffamily
%   \frenchspacing

%   Some journals require a graphical entry for the Table of Contents. This
%   should be laid out ``print ready'' so that the sizing of the text is correct.

%   The space available depends on the journal: J. Am. Chem. Soc. allows 3.25 in
%   by 1.75 in and requires sanserif text. Some journals want different sizes:
%   you can easily adjust here.
  
%   The two rules either side of the content are there to help judge the height
%   of your material: they may be deleted once not required.
  
% \end{minipage}%
% \rule{0.05in}{1.75in}

\end{document}
