%%%%%%%%%%%%%%%%%%%%%%%%%%%%%%%%%%%%%%%%%%%%%%%%%%%%%%%%%%%%%%%%%%%%%
%% This is a "template" model document for submission to the
%% American Chemical Society (ACS).
%%
%% The guidance here contains information about how you may wish to
%% modify it to match the requirements of various ACS journals. The
%% ACS do *not* typeset accepted articles using LaTeX, so there is
%% no specific class required.
%%
%% This template deliberately does *not* seek to reproduce
%% the layout of the typeset journal: this is explicitly not
%% required by the ACS for LaTeX submissions.
%%
%% Please report any issues with the template at
%% https://github.com/josephwright/acs-template/issues
%%
%% Released under the Creative Commons 0 license
%% https://creativecommons.org/public-domain/cc0/
%% 
%% Copyight (c) 2025 Joseph Wright
%%%%%%%%%%%%%%%%%%%%%%%%%%%%%%%%%%%%%%%%%%%%%%%%%%%%%%%%%%%%%%%%%%%%%
\documentclass[letterpaper]{article}

%%%%%%%%%%%%%%%%%%%%%%%%%%%%%%%%%%%%%%%%%%%%%%%%%%%%%%%%%%%%%%%%%%%%%
%% Font setup - delete if you are using LuaLaTeX
%%%%%%%%%%%%%%%%%%%%%%%%%%%%%%%%%%%%%%%%%%%%%%%%%%%%%%%%%%%%%%%%%%%%%
\usepackage[T1]{fontenc}

%%%%%%%%%%%%%%%%%%%%%%%%%%%%%%%%%%%%%%%%%%%%%%%%%%%%%%%%%%%%%%%%%%%%%
%% Adjust the margins and allow for line spacing
%%%%%%%%%%%%%%%%%%%%%%%%%%%%%%%%%%%%%%%%%%%%%%%%%%%%%%%%%%%%%%%%%%%%%
\usepackage{geometry}
\geometry{margin = 1in}
\usepackage{setspace}

%%%%%%%%%%%%%%%%%%%%%%%%%%%%%%%%%%%%%%%%%%%%%%%%%%%%%%%%%%%%%%%%%%%%%
%% Reference support
%%
%% The recommended method for producing the reference section is
%% to use biblatex. If you wish to use a classical BibTeX
%% approach, this is easiest to achieve using the achemso package.
%% In that case, you should remove the biblatex lines.
%%%%%%%%%%%%%%%%%%%%%%%%%%%%%%%%%%%%%%%%%%%%%%%%%%%%%%%%%%%%%%%%%%%%%
% You can adjust the printing of DOI, article title, etc. using
% package options, e.g. "doi = true"; see the biblatex manual for
% details of adjusting the number of authors printed, e.g.
% "maxnames = 15" to print no more than 15 authors.
\usepackage[style = chem-acs, articletitle = true]{biblatex}
\DeclareFieldFormat{journaltitle}{\mkbibemph{\printfield{shortjournal}}}
\addbibresource{reference.bib}
% If you are using classical BibTeX, remove the above lines and 
% uncomment:
%\usepackage{achemso}

%%%%%%%%%%%%%%%%%%%%%%%%%%%%%%%%%%%%%%%%%%%%%%%%%%%%%%%%%%%%%%%%%%%%%
%% Graphic inclusion and scheme and chart support
%%%%%%%%%%%%%%%%%%%%%%%%%%%%%%%%%%%%%%%%%%%%%%%%%%%%%%%%%%%%%%%%%%%%%
\usepackage{graphicx}
\usepackage{float}
\newfloat{scheme}{htbp}{los}
\floatname{scheme}{Scheme}
\floatname{chart}{Chart}
\newfloat{graph}{htbp}{loh}

%%%%%%%%%%%%%%%%%%%%%%%%%%%%%%%%%%%%%%%%%%%%%%%%%%%%%%%%%%%%%%%%%%%%%
%% Common support packages
%%%%%%%%%%%%%%%%%%%%%%%%%%%%%%%%%%%%%%%%%%%%%%%%%%%%%%%%%%%%%%%%%%%%%
\usepackage{chemformula} % Formulas using \ch{}
% or
\usepackage[version = 4]{mhchem} % Formulas using \ce{}

%%%%%%%%%%%%%%%%%%%%%%%%%%%%%%%%%%%%%%%%%%%%%%%%%%%%%%%%%%%%%%%%%%%%%
%% Many journals require that sections are unnumbered: this 
%% is activated here
%%%%%%%%%%%%%%%%%%%%%%%%%%%%%%%%%%%%%%%%%%%%%%%%%%%%%%%%%%%%%%%%%%%%%
\setcounter{secnumdepth}{-1}

%%%%%%%%%%%%%%%%%%%%%%%%%%%%%%%%%%%%%%%%%%%%%%%%%%%%%%%%%%%%%%%%%%%%%
%% Place any additional macros here.  Please use \newcommand* where
%% possible, and avoid layout-changing macros (which are not used
%% when typesetting).
%%%%%%%%%%%%%%%%%%%%%%%%%%%%%%%%%%%%%%%%%%%%%%%%%%%%%%%%%%%%%%%%%%%%%
\newcommand*\mycommand[1]{\texttt{\emph{#1}}}
\newcommand{\etal}{{\it et al\/}\ }
%%%%%%%%%%%%%%%%%%%%%%%%%%%%%%%%%%%%%%%%%%%%%%%%%%%%%%%%%%%%%%%%%%%%%
%% Author and title data:
%% the authblk package is currently the simplest way to provide this
%%%%%%%%%%%%%%%%%%%%%%%%%%%%%%%%%%%%%%%%%%%%%%%%%%%%%%%%%%%%%%%%%%%%%
\usepackage{authblk}
\author[1,2]{Yuta Goto}
\author[1]{Ryo Matsumura*}
\author[1,2]{Fukata Naoki*}
\affil[1]{International Center for Materials Nanoarchitectonics (MANA), National Institute for Materials Science, 1-1 Namiki, Tsukuba, Ibaraki, 305-0044, Japan}
\affil[2]{Graduate School of Science and Technology, University of Tsukuba, 1-1-1 Tennodai, Tsukuba, Ibaraki, 305-8573, Japan}

\title{Optimal conditions for crystallization of amorphous germanium thin films by high-speed laser annealing}
% Use the \date command for email address(s) of corresponding authors
\date{*Email: MATSUMURA.Ryo@nims.go.jp, FUKATA.Naoki@nims.go.jp}

\begin{document}

\maketitle

\begin{abstract}
\end{abstract}

\section*{Keywords}
thin film, germanium, laser anneal, crystallization

%%%%%%%%%%%%%%%%%%%%%%%%%%%%%%%%%%%%%%%%%%%%%%%%%%%%%%%%%%%%%%%%%%%%%
%% Start the main part of the manuscript here.
%%%%%%%%%%%%%%%%%%%%%%%%%%%%%%%%%%%%%%%%%%%%%%%%%%%%%%%%%%%%%%%%%%%%%
\section{Introduction}
Germanium (Ge) exhibits superior carrier mobility compared to silicon (Si), making it attractive for electronic applications.
To realize these applications, Ge-on-insulator (GOI) structures are essential not only because Ge substrates are expensive but also the parasitic resistance and the carrier mobility can be improved.
The easiest way to fabricate GOI structures is crystallization of amorphous Ge (a-Ge) thin films deposited on insulator substrates, such as SiO$_2$/Si.
Solid phase crystallization (SPC) is a widely used method for crystallization of a-Ge thin films, which can be performed just by thermal annealing at 400 - 500 ${}^\circ$C.
Although it can crystallize a-Ge films in large area, it requires long annealing time up to 100 hours.
In addition, the crystal orientation of the film is random, which limits the performance of devices fabricated on the films.
Metal-induced crystallization (MIC) also can crystallize a-Ge films at low temperature, but the metal contamination from the catalyst metals is a critical issue for device applications. 
Liquid phase crystallization (LPC) is another method for crystallization of a-Ge films.
Although many reports exist on LPC of a-Ge films using rapid thermal annealing (RTA) or pulsed laser annealing, continuous wave laser annealing has attracted attention as it can achieve large-area single-crystal films by scanning the laser beam.
Not only that, it can be utilized to activate dopants or alloy elements in Ge films beyond the equilibrium solubility limit as the annealing time can be modulated from miliseconds to microseconds by the laser scan speed.
In addition, various wavelengths of lasers, such as visible and ultraviolet, enabling selective heating of Ge films without heating the substrates.
However, continuous wave laser annealing possesses many parameters originating from laser (laser power, scan speed), Ge film (deposition temperature), and capping layer (thickness).
To achieve high-quality crystallization of a-Ge films by LPC, understanding effects of those parameters on the crystallization behavior is essential. 
Thus, in this study, we systematically investigated the effects of various parameters on the crystallization behavior of a-Ge films by LPC.
Whereas LPC melts amorphous films and solidifies them into crystalline films, it was found that the initial properties of the films affect the crystallization behavior. 
Although some reports achieved large-area single-crystal using continous laser scan, the scan speed was slow (- XX cm/s) and not suitable for mass production.
In addition, required grain size for thin film transistors is up to several micrometers, thus deviation inside the thin film is more important for device performance.
Pulsed laser annealing, whose annealing time is on the order of nanoseconds, is generally employed for crystallization of amorohous silicon to avoid thermal damage to the substrate.
However, this limits the crystal size up to a few micrometers.
Moreover, it is difficult to irradiate large are of a-Ge films because Ge has high surface energy and it agglomerates during pulsed laser annealing.
Thus, we employed high-speed continuous wave laser annealing whose annealing time is on the order of microseconds, to achieve fast crystallization of a-Ge films without thermal damage and agglomeration.
Finite element method (FEM) simulation was also performed to understand the temperature evolution during laser annealing and the crystallization mechanism change when the parameters were swept.
In addition, we combined post annealing with continuous wave laser annealing to further improve the crystal quality and the carrier mobility of the films.
Defects in Ge films are electrically active and they have p-type conductivity.
We demonstrated that the carrier concentration decreased to $\sim$ 1.0 $\times$ 10$^{17}$ cm$^{-3}$ with the Hall mobility up to 400 cm$^2$/Vs.
The obtained results offer deep insights into the crystallization of a-Ge films by high-speed laser annealing and pave a way for fabricating high-performance Ge-based devices on insulator substrates.

\section{Experimental methods}
\subsection{Sample preparation}
Quartz glass substrates were ultrasonically cleaned by accetone, deionized water.
Subsequently, they were cleaned by dilluted hydrofluoric acid. 
Amorphous Ge thin films with a thickness of 100 nm were deposited on the substrates by molecular beam deposition.
Capping layers of SiO$_2$ were deposited on the a-Ge films by RF magnetron sputtering to suppress the agglomeration and oxidation of the films.
High-speed laser annealing was performed using a continous waeve laser with a wavelemgth of 532 nm.
The samples were mounted on a rotable stage and the laser beam was scanned at various speeds from XX to 15 m/s.
The laser possessed a Gaussian profile with a diamter of 20 \textmu m.
After annealing, the capping layers were removed by hydrofluoric acid for measurements.
For Hall measurements, the annealed samples were patterned into clover-leaf structures by photolithography and dry etching.
Post annealing was also performed in a nitrogen atmosphere at 500 ${}^\circ$C to investigate its effect on the crystallinity and carrier mobility.
\subsection{Characterization}
Electron backscatter diffraction (EBSD) measurements were performed to evaluate the crystal orientation and grain size of the annealed Ge films.
Raman spectroscopy was performed to evaluate the crystallinity of the deposited Ge films before laser annealing.
Ultravoilet-Visible (UV-Vis) spectroscopy was performed to evaluate the absorption properties of the deposited Ge films.
Hall measurements were conducted using the van der Pauw method to evaluate the carrier concentration and Hall mobility of the annealed Ge films at room temperature.

\section{Results and discussion}
%%%%%% Section : FIG1 %%%%%%
\begin{figure}[htbp]
  \centering
  \includegraphics[width=\linewidth]{../Figures/Figure1/Figure1.pdf}
  \caption{IPF colors of the laser-annealed Ge films deposited at (a) room temperature, (b) 150 ${}^\circ$C, (c) 300 ${}^\circ$C, (d) 450 ${}^\circ$C, and (e) 600 ${}^\circ$C, respectively. The scan speed and laser power were fixed at 15 m/s and 2000 mW, respectively. Schematic illustrations of the crystallization behavior of the Ge films deposited at (f) low temperature below 150 ${}^\circ$C and (g) high temperature above 300 ${}^\circ$C during laser annealing.}
  \centering
  \includegraphics[width=\linewidth]{../Figures/Figure2/Figure2.pdf}
  \caption{IPF colors of the laser-annealed Ge films deposited at (a) room temperature, (b) 150 ${}^\circ$C, (c) 300 ${}^\circ$C, (d) 450 ${}^\circ$C, and (e) 600 ${}^\circ$C, respectively. The scan speed and laser power were fixed at 0.67 m/s and 300 mW, respectively.}
  \label{fgr:fig2}
\end{figure}
%%%%%% Section : FIG3 %%%%%%
\begin{figure}[htbp]
  \centering
  \includegraphics[width=\linewidth]{../Figures/Figure3/Figure3.pdf}
  \caption{(a) Raman spectra of the Ge films deposited at room temperature, 150 ${}^\circ$C, 300 ${}^\circ$C, 450 ${}^\circ$C, and 600 ${}^\circ$C, respectively. The dashed line indicates the vibrational mode of Ge obtained from Ge substrate (300.2 cm$^{-1}$). (b) Absorption spectra of the Ge films deposited at room temperature, 150 ${}^\circ$C, 300 ${}^\circ$C, 450 ${}^\circ$C, and 600 ${}^\circ$C, respectively. (b-1) Transmittance spectra of the Ge films deposited at room temperature, 150 ${}^\circ$C, 300 ${}^\circ$C, 450 ${}^\circ$C, and 600 ${}^\circ$C, respectively. (b-2) Absorption spectra and (b-2) absorbance spectra of the Ge films deposited at room temperature, 150 ${}^\circ$C, 300 ${}^\circ$C, 450 ${}^\circ$C, and 600 ${}^\circ$C, respectively. The calculated transmittance and absorbance spectra of a single-crystal Ge film on quartz glass substrate are also shown for comparison. The dashed line indicates the wavelength of the laser (532 nm).}
  \label{fgr:fig3}
\end{figure}
%%%%%% Section : FIG4 %%%%%%
\begin{figure}[htbp]
  \centering
  \includegraphics[width=\linewidth]{../Figures/Figure4/Figure4.pdf}
  \caption{IPF colors of the laser-annealed Ge films deposited at 450 ${}^\circ$C with various scan speeds of XXX.}
  \label{fgr:fig4}
\end{figure}
%%%%%% Section : FIG5 %%%%%%
% \begin{figure}[htbp]
%   \centering
%   \includegraphics[width=\linewidth]{../Figures/Figure5/Figure5.pdf}
%   \caption{}
%   \label{fgr:fig5}
% \end{figure}




\section*{Acknowledgements}
This research was financially supported by JSPS Kakenhi (grant No.23K13370), and the World Premier International Research Center Initiative (WPI-Initiative).
This work was also partially supported by "Advanced Research Infrastructure for Materials and Nanotechnology in Japan (ARIM)", MEXT, Japan, proposal number.

\section*{Supporting information}
The following files are available free of charge.
\begin{itemize}
  \item support\_info.pdf: A schematic illustration of the clover-leaf structure of the annealed Ge films by photolithography and dry etching (Figure S1), Description of equations and material parameters used in the FEM simulation (Table S1 and Figure S2).   
\end{itemize}

\printbibliography
%\bibliography{acs-template.bib}

\newpage

% \rule{0.05in}{1.75in}%
% \begin{minipage}[b][1.75in]{3.25in}
%   \sffamily
%   \frenchspacing

%   Some journals require a graphical entry for the Table of Contents. This
%   should be laid out ``print ready'' so that the sizing of the text is correct.

%   The space available depends on the journal: J. Am. Chem. Soc. allows 3.25 in
%   by 1.75 in and requires sanserif text. Some journals want different sizes:
%   you can easily adjust here.
  
%   The two rules either side of the content are there to help judge the height
%   of your material: they may be deleted once not required.
  
% \end{minipage}%
% \rule{0.05in}{1.75in}

\end{document}
